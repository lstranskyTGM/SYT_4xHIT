% Options for packages loaded elsewhere
\PassOptionsToPackage{unicode}{hyperref}
\PassOptionsToPackage{hyphens}{url}
%
\documentclass[
]{article}
\usepackage{amsmath,amssymb}
\usepackage{iftex}
\ifPDFTeX
  \usepackage[T1]{fontenc}
  \usepackage[utf8]{inputenc}
  \usepackage{textcomp} % provide euro and other symbols
\else % if luatex or xetex
  \usepackage{unicode-math} % this also loads fontspec
  \defaultfontfeatures{Scale=MatchLowercase}
  \defaultfontfeatures[\rmfamily]{Ligatures=TeX,Scale=1}
\fi
\usepackage{lmodern}
\ifPDFTeX\else
  % xetex/luatex font selection
\fi
% Use upquote if available, for straight quotes in verbatim environments
\IfFileExists{upquote.sty}{\usepackage{upquote}}{}
\IfFileExists{microtype.sty}{% use microtype if available
  \usepackage[]{microtype}
  \UseMicrotypeSet[protrusion]{basicmath} % disable protrusion for tt fonts
}{}
\makeatletter
\@ifundefined{KOMAClassName}{% if non-KOMA class
  \IfFileExists{parskip.sty}{%
    \usepackage{parskip}
  }{% else
    \setlength{\parindent}{0pt}
    \setlength{\parskip}{6pt plus 2pt minus 1pt}}
}{% if KOMA class
  \KOMAoptions{parskip=half}}
\makeatother
\usepackage{xcolor}
\usepackage[margin=1in]{geometry}
\usepackage{color}
\usepackage{fancyvrb}
\newcommand{\VerbBar}{|}
\newcommand{\VERB}{\Verb[commandchars=\\\{\}]}
\DefineVerbatimEnvironment{Highlighting}{Verbatim}{commandchars=\\\{\}}
% Add ',fontsize=\small' for more characters per line
\usepackage{framed}
\definecolor{shadecolor}{RGB}{248,248,248}
\newenvironment{Shaded}{\begin{snugshade}}{\end{snugshade}}
\newcommand{\AlertTok}[1]{\textcolor[rgb]{0.94,0.16,0.16}{#1}}
\newcommand{\AnnotationTok}[1]{\textcolor[rgb]{0.56,0.35,0.01}{\textbf{\textit{#1}}}}
\newcommand{\AttributeTok}[1]{\textcolor[rgb]{0.13,0.29,0.53}{#1}}
\newcommand{\BaseNTok}[1]{\textcolor[rgb]{0.00,0.00,0.81}{#1}}
\newcommand{\BuiltInTok}[1]{#1}
\newcommand{\CharTok}[1]{\textcolor[rgb]{0.31,0.60,0.02}{#1}}
\newcommand{\CommentTok}[1]{\textcolor[rgb]{0.56,0.35,0.01}{\textit{#1}}}
\newcommand{\CommentVarTok}[1]{\textcolor[rgb]{0.56,0.35,0.01}{\textbf{\textit{#1}}}}
\newcommand{\ConstantTok}[1]{\textcolor[rgb]{0.56,0.35,0.01}{#1}}
\newcommand{\ControlFlowTok}[1]{\textcolor[rgb]{0.13,0.29,0.53}{\textbf{#1}}}
\newcommand{\DataTypeTok}[1]{\textcolor[rgb]{0.13,0.29,0.53}{#1}}
\newcommand{\DecValTok}[1]{\textcolor[rgb]{0.00,0.00,0.81}{#1}}
\newcommand{\DocumentationTok}[1]{\textcolor[rgb]{0.56,0.35,0.01}{\textbf{\textit{#1}}}}
\newcommand{\ErrorTok}[1]{\textcolor[rgb]{0.64,0.00,0.00}{\textbf{#1}}}
\newcommand{\ExtensionTok}[1]{#1}
\newcommand{\FloatTok}[1]{\textcolor[rgb]{0.00,0.00,0.81}{#1}}
\newcommand{\FunctionTok}[1]{\textcolor[rgb]{0.13,0.29,0.53}{\textbf{#1}}}
\newcommand{\ImportTok}[1]{#1}
\newcommand{\InformationTok}[1]{\textcolor[rgb]{0.56,0.35,0.01}{\textbf{\textit{#1}}}}
\newcommand{\KeywordTok}[1]{\textcolor[rgb]{0.13,0.29,0.53}{\textbf{#1}}}
\newcommand{\NormalTok}[1]{#1}
\newcommand{\OperatorTok}[1]{\textcolor[rgb]{0.81,0.36,0.00}{\textbf{#1}}}
\newcommand{\OtherTok}[1]{\textcolor[rgb]{0.56,0.35,0.01}{#1}}
\newcommand{\PreprocessorTok}[1]{\textcolor[rgb]{0.56,0.35,0.01}{\textit{#1}}}
\newcommand{\RegionMarkerTok}[1]{#1}
\newcommand{\SpecialCharTok}[1]{\textcolor[rgb]{0.81,0.36,0.00}{\textbf{#1}}}
\newcommand{\SpecialStringTok}[1]{\textcolor[rgb]{0.31,0.60,0.02}{#1}}
\newcommand{\StringTok}[1]{\textcolor[rgb]{0.31,0.60,0.02}{#1}}
\newcommand{\VariableTok}[1]{\textcolor[rgb]{0.00,0.00,0.00}{#1}}
\newcommand{\VerbatimStringTok}[1]{\textcolor[rgb]{0.31,0.60,0.02}{#1}}
\newcommand{\WarningTok}[1]{\textcolor[rgb]{0.56,0.35,0.01}{\textbf{\textit{#1}}}}
\usepackage{graphicx}
\makeatletter
\def\maxwidth{\ifdim\Gin@nat@width>\linewidth\linewidth\else\Gin@nat@width\fi}
\def\maxheight{\ifdim\Gin@nat@height>\textheight\textheight\else\Gin@nat@height\fi}
\makeatother
% Scale images if necessary, so that they will not overflow the page
% margins by default, and it is still possible to overwrite the defaults
% using explicit options in \includegraphics[width, height, ...]{}
\setkeys{Gin}{width=\maxwidth,height=\maxheight,keepaspectratio}
% Set default figure placement to htbp
\makeatletter
\def\fps@figure{htbp}
\makeatother
\setlength{\emergencystretch}{3em} % prevent overfull lines
\providecommand{\tightlist}{%
  \setlength{\itemsep}{0pt}\setlength{\parskip}{0pt}}
\setcounter{secnumdepth}{-\maxdimen} % remove section numbering
\ifLuaTeX
  \usepackage{selnolig}  % disable illegal ligatures
\fi
\IfFileExists{bookmark.sty}{\usepackage{bookmark}}{\usepackage{hyperref}}
\IfFileExists{xurl.sty}{\usepackage{xurl}}{} % add URL line breaks if available
\urlstyle{same}
\hypersetup{
  pdftitle={DSAI\_M1},
  pdfauthor={Leonhard Stransky},
  hidelinks,
  pdfcreator={LaTeX via pandoc}}

\title{DSAI\_M1}
\author{Leonhard Stransky}
\date{2023-09-19}

\begin{document}
\maketitle

\hypertarget{data-science-einfuxfchrung-in-relevante-softwareumgebungen}{%
\section{Data Science ``Einführung in relevante
Softwareumgebungen'\,'}\label{data-science-einfuxfchrung-in-relevante-softwareumgebungen}}

\hypertarget{einfuxfchrung}{%
\subsection{Einführung:}\label{einfuxfchrung}}

Die Theorieeinheiten sollen als Hintergrundinformation zur Durchführung
einer eigenen Analyse dienen. Zum Umgang mit R und RStudio oder Python
ist der Link aus den Theorie Einheiten empfohlen. Als Vorlage für die
Laborprotokolle steht euch im Kurs ein Template für R Markdown zur
Verfügung.

\hypertarget{ziele}{%
\subsection{Ziele:}\label{ziele}}

Das Ziel ist es, \href{https://learn.datacamp.com/}{DataCamp} als
Anleitung und R als Werkzeug zur Datenexploration und Visualisierung
selbständig einsetzen und erste Ergebnisse interpretieren zu können.

\hypertarget{aufgabe-1gk}{%
\subsection{Aufgabe 1(GK):}\label{aufgabe-1gk}}

Erarbeite dir durch den Kurs
\href{https://methodenlehre.github.io/einfuehrung-in-R/}{Einführung in
R} in den Kapiteln 2 bis 4 die Grundlagen der Sprache Programmiersprache
R und vertiefe dich dann in Visualisierung mit der Library `ggplot' in
Kapitel 5. Fasse zusätzlich die Inhalte von
\href{https://www.w3schools.com/r/}{w3schools} der Kapitel ``Tutorial''
und ``Data Structures'' (GK)

Führe dabei ein Protokoll in einem Markdown File über Informationen,
Dokumentation und Codes als Nachschlagemöglichkeit, da dir der Kurs nur
für 6 Monate zur Verfügung stehen wird.

\pagebreak

\hypertarget{die-r-sprache}{%
\subsection{2. Die R Sprache:}\label{die-r-sprache}}

\hypertarget{uxfcbung1}{%
\subsubsection{Übung1:}\label{uxfcbung1}}

\begin{Shaded}
\begin{Highlighting}[]
\DecValTok{1}\SpecialCharTok{/}\DecValTok{3} \SpecialCharTok{*}\NormalTok{ ((}\DecValTok{1}\SpecialCharTok{+}\DecValTok{3}\SpecialCharTok{+}\DecValTok{5}\SpecialCharTok{+}\DecValTok{7}\SpecialCharTok{+}\DecValTok{2}\NormalTok{)}\SpecialCharTok{/}\NormalTok{(}\DecValTok{3}\SpecialCharTok{+}\DecValTok{5}\SpecialCharTok{+}\DecValTok{4}\NormalTok{))}
\end{Highlighting}
\end{Shaded}

\begin{verbatim}
## [1] 0.5
\end{verbatim}

\begin{Shaded}
\begin{Highlighting}[]
\FunctionTok{exp}\NormalTok{(}\DecValTok{1}\NormalTok{)}
\end{Highlighting}
\end{Shaded}

\begin{verbatim}
## [1] 2.718282
\end{verbatim}

\begin{Shaded}
\begin{Highlighting}[]
\FunctionTok{sqrt}\NormalTok{(}\DecValTok{2}\NormalTok{)}
\end{Highlighting}
\end{Shaded}

\begin{verbatim}
## [1] 1.414214
\end{verbatim}

\begin{Shaded}
\begin{Highlighting}[]
\DecValTok{8}\SpecialCharTok{\^{}}\NormalTok{(}\DecValTok{1}\SpecialCharTok{/}\DecValTok{3}\NormalTok{)}
\end{Highlighting}
\end{Shaded}

\begin{verbatim}
## [1] 2
\end{verbatim}

\begin{Shaded}
\begin{Highlighting}[]
\FunctionTok{log2}\NormalTok{(}\DecValTok{8}\NormalTok{)}
\end{Highlighting}
\end{Shaded}

\begin{verbatim}
## [1] 3
\end{verbatim}

\hypertarget{uxfcbung2}{%
\subsubsection{Übung2:}\label{uxfcbung2}}

\begin{enumerate}
\def\labelenumi{\arabic{enumi}.}
\tightlist
\item
  Erzeugen Sie eine Sequenz von 0 bis 100 in 5-er Schritten.
\item
  Berechnen Sie den Mittelwert des Vektors {[}1, 3, 4, 7, 11, 2{]}.
\item
  Lassen Sie sich die Spannweite x(max)-x(min) dieses Vektors ausgeben.
\item
  Berechnen Sie die Summe dieses Vektors.
\item
  Zentrieren Sie diesen Vektor.
\item
  Simulieren Sie einen Münzwurf mit der Funktion sample(). Tipp: nehmen
  Sie für Kopf 1 und für Zahl 0. Simulieren Sie 100 Münzwürfe.
\item
  Simulieren Sie eine Trick-Münze mit P!=0.5
\item
  Generieren Sie einen Vektor, der aus 100 Wiederholungen der Zahl 3
  besteht.
\end{enumerate}

\begin{Shaded}
\begin{Highlighting}[]
\FunctionTok{seq}\NormalTok{(}\AttributeTok{from =} \DecValTok{0}\NormalTok{, }\AttributeTok{to =} \DecValTok{100}\NormalTok{, }\AttributeTok{by =} \DecValTok{5}\NormalTok{)}
\end{Highlighting}
\end{Shaded}

\begin{verbatim}
##  [1]   0   5  10  15  20  25  30  35  40  45  50  55  60  65  70  75  80  85  90
## [20]  95 100
\end{verbatim}

\begin{Shaded}
\begin{Highlighting}[]
\FunctionTok{mean}\NormalTok{(}\FunctionTok{c}\NormalTok{(}\DecValTok{1}\NormalTok{,}\DecValTok{3}\NormalTok{,}\DecValTok{4}\NormalTok{,}\DecValTok{7}\NormalTok{,}\DecValTok{11}\NormalTok{,}\DecValTok{2}\NormalTok{))}
\end{Highlighting}
\end{Shaded}

\begin{verbatim}
## [1] 4.666667
\end{verbatim}

\begin{Shaded}
\begin{Highlighting}[]
\NormalTok{b }\OtherTok{\textless{}{-}} \FunctionTok{range}\NormalTok{(}\FunctionTok{c}\NormalTok{(}\DecValTok{1}\NormalTok{,}\DecValTok{3}\NormalTok{,}\DecValTok{4}\NormalTok{,}\DecValTok{7}\NormalTok{,}\DecValTok{11}\NormalTok{,}\DecValTok{2}\NormalTok{))}
\NormalTok{b[}\DecValTok{2}\NormalTok{] }\SpecialCharTok{{-}}\NormalTok{ b[}\DecValTok{1}\NormalTok{]}
\end{Highlighting}
\end{Shaded}

\begin{verbatim}
## [1] 10
\end{verbatim}

\begin{Shaded}
\begin{Highlighting}[]
\FunctionTok{sum}\NormalTok{(}\FunctionTok{c}\NormalTok{(}\DecValTok{1}\NormalTok{,}\DecValTok{3}\NormalTok{,}\DecValTok{4}\NormalTok{,}\DecValTok{7}\NormalTok{,}\DecValTok{11}\NormalTok{,}\DecValTok{2}\NormalTok{))}
\end{Highlighting}
\end{Shaded}

\begin{verbatim}
## [1] 28
\end{verbatim}

\begin{Shaded}
\begin{Highlighting}[]
\FunctionTok{scale}\NormalTok{(}\FunctionTok{c}\NormalTok{(}\DecValTok{1}\NormalTok{,}\DecValTok{3}\NormalTok{,}\DecValTok{4}\NormalTok{,}\DecValTok{7}\NormalTok{,}\DecValTok{11}\NormalTok{,}\DecValTok{2}\NormalTok{), }\AttributeTok{center =} \ConstantTok{TRUE}\NormalTok{, }\AttributeTok{scale =} \ConstantTok{FALSE}\NormalTok{)}
\end{Highlighting}
\end{Shaded}

\begin{verbatim}
##            [,1]
## [1,] -3.6666667
## [2,] -1.6666667
## [3,] -0.6666667
## [4,]  2.3333333
## [5,]  6.3333333
## [6,] -2.6666667
## attr(,"scaled:center")
## [1] 4.666667
\end{verbatim}

\begin{Shaded}
\begin{Highlighting}[]
\FunctionTok{sample}\NormalTok{(}\FunctionTok{c}\NormalTok{(}\DecValTok{0}\NormalTok{,}\DecValTok{1}\NormalTok{), }\AttributeTok{size =} \DecValTok{100}\NormalTok{, }\AttributeTok{replace =} \ConstantTok{TRUE}\NormalTok{)}
\end{Highlighting}
\end{Shaded}

\begin{verbatim}
##   [1] 1 0 1 1 0 0 0 0 0 1 0 0 1 0 1 0 0 0 1 0 1 1 0 1 1 1 1 1 1 1 0 0 1 0 1 0 1
##  [38] 1 1 0 0 0 0 0 1 0 0 1 0 1 0 1 0 1 1 1 0 0 0 1 0 1 1 1 1 1 1 1 1 0 0 0 0 0
##  [75] 0 0 0 1 1 1 1 1 1 0 0 1 1 0 0 0 0 0 0 1 1 0 0 1 0 1
\end{verbatim}

\begin{Shaded}
\begin{Highlighting}[]
\FunctionTok{sample}\NormalTok{(}\FunctionTok{c}\NormalTok{(}\DecValTok{0}\NormalTok{,}\DecValTok{1}\NormalTok{,}\DecValTok{1}\NormalTok{,}\DecValTok{1}\NormalTok{,}\DecValTok{1}\NormalTok{), }\AttributeTok{size =} \DecValTok{100}\NormalTok{, }\AttributeTok{replace =} \ConstantTok{TRUE}\NormalTok{)}
\end{Highlighting}
\end{Shaded}

\begin{verbatim}
##   [1] 1 1 0 1 1 1 1 1 0 1 1 1 1 1 1 1 1 1 0 1 1 1 1 1 1 1 1 1 1 1 1 1 1 0 1 1 1
##  [38] 1 1 1 1 1 0 1 1 1 1 0 0 1 1 1 0 1 1 1 0 1 0 1 1 1 1 1 1 1 0 1 1 1 1 1 1 0
##  [75] 0 1 0 0 1 1 1 0 0 1 1 1 1 1 1 1 1 1 1 0 1 0 1 1 0 1
\end{verbatim}

\begin{Shaded}
\begin{Highlighting}[]
\FunctionTok{rep}\NormalTok{(}\DecValTok{3}\NormalTok{, }\AttributeTok{times =} \DecValTok{100}\NormalTok{)}
\end{Highlighting}
\end{Shaded}

\begin{verbatim}
##   [1] 3 3 3 3 3 3 3 3 3 3 3 3 3 3 3 3 3 3 3 3 3 3 3 3 3 3 3 3 3 3 3 3 3 3 3 3 3
##  [38] 3 3 3 3 3 3 3 3 3 3 3 3 3 3 3 3 3 3 3 3 3 3 3 3 3 3 3 3 3 3 3 3 3 3 3 3 3
##  [75] 3 3 3 3 3 3 3 3 3 3 3 3 3 3 3 3 3 3 3 3 3 3 3 3 3 3
\end{verbatim}

\hypertarget{uxfcbung3}{%
\subsubsection{Übung3:}\label{uxfcbung3}}

Wiederholen Sie einige der Beispiele von oben, aber speichern Sie
diesmal die Resultate in neuen Variablen ab. Sie können die Variablen
dann für weitere Operationen wieder verwenden.

\begin{verbatim}
## [1] 4.666667
\end{verbatim}

\begin{verbatim}
## [1] 4.7
\end{verbatim}

\hypertarget{uxfcbung4}{%
\subsubsection{Übung4:}\label{uxfcbung4}}

Tippen Sie in der Konsole scale( und drücken Sie TAB. Sie sehen, dass
diese Funktion drei Argumente hat, x, center und scale. Schreiben Sie
scale(vektor, scale = gefolgt von TAB. Sie sehen, dass scale = TRUE den
default Wert TRUE hat.

Was sind die Argumente der Funktion round()? Hat eines der Argumente
einen default Wert?

Schauen Sie im Help Viewer nach, was die Funktion rnorm() macht. Was
sind die Argumente. Was bedeuten die default Werte?

Schauen Sie im Help Viewer nach, welche Argumente die seq() Funktion
hat.

Was machen folgende Funktionsaufrufe?

\begin{verbatim}
## [1] 1
\end{verbatim}

\begin{verbatim}
##  [1]  1  2  3  4  5  6  7  8  9 10
\end{verbatim}

\begin{verbatim}
## [1] 1 3 5 7 9
\end{verbatim}

\begin{verbatim}
##  [1]  1.000000  1.473684  1.947368  2.421053  2.894737  3.368421  3.842105
##  [8]  4.315789  4.789474  5.263158  5.736842  6.210526  6.684211  7.157895
## [15]  7.631579  8.105263  8.578947  9.052632  9.526316 10.000000
\end{verbatim}

round:

Bsp: round(x, digits = 1)

x ist die Zahl oder der Vektor

digits ist die Anzahl der Dezimalstellen auf die gerundet werden soll

Die Verwendung von Digits ist optional und hat einen default wert von 0.

rnorm: Dies wird verwendet um Zufallszahlen aus einer Normalverteilung
zu generieren.

Bsp: rnorm(n, mean = 0, sd = 1)

n ist die Anzahl der Zufallszahlen. (erfordert)

mean ist der Durchschnitt (Mittelwert) der Normalverteilung.

sd ist die Standardabweichung der Normalverteilung.

mean und sd sind optionale Argumente. mean hat als default wert 0 sund
sd 1 (Dies enstpspricht einer Standardnormalverteilung)

seq: Dies wird verwendet um reguläre Sequenzen von Zahlen zu generieren.

Bsp: seq()´

seq(from, to, by, length.out, along.with, \ldots)

from ist der Startwert der Sequenz. (optional) (default wert: 1)

to: Der Endwert der Sequenz. (optional) (default wert: 1)

by: Der Inkrementwert, um den die Sequenz erhöht wird. (optional)
default wert wird automatisch berechnet, um die gewünschte Länge der
Sequenz length.out zu erreichen.

length.out: Die gewünschte Länge der Sequenz. (optional) default wert
ist NULL. Wenn length.out nicht angegeben wird, wird die Länge der
Sequenz basierend auf den anderen Argumenten berechnet.

along.with: Übernimmt die Länge der Sequenz von einem anderen Vektor
oder einer anderen Sequenz. (optional)

\hypertarget{character-vektors}{%
\subsubsection{Character Vektors}\label{character-vektors}}

\begin{Shaded}
\begin{Highlighting}[]
\NormalTok{letters[}\DecValTok{1}\SpecialCharTok{:}\DecValTok{3}\NormalTok{]}
\end{Highlighting}
\end{Shaded}

\begin{verbatim}
## [1] "a" "b" "c"
\end{verbatim}

\begin{Shaded}
\begin{Highlighting}[]
\NormalTok{text }\OtherTok{\textless{}{-}} \FunctionTok{c}\NormalTok{(}\StringTok{"these are"}\NormalTok{, }\StringTok{"some strings"}\NormalTok{)}
\NormalTok{text}
\end{Highlighting}
\end{Shaded}

\begin{verbatim}
## [1] "these are"    "some strings"
\end{verbatim}

\begin{Shaded}
\begin{Highlighting}[]
\FunctionTok{length}\NormalTok{(text)}
\end{Highlighting}
\end{Shaded}

\begin{verbatim}
## [1] 2
\end{verbatim}

\begin{Shaded}
\begin{Highlighting}[]
\NormalTok{vorname }\OtherTok{\textless{}{-}} \StringTok{"Ronald Aylmer"}
\NormalTok{nachname }\OtherTok{\textless{}{-}} \StringTok{"Fisher"}
\FunctionTok{paste}\NormalTok{(}\StringTok{"Mein Name ist:"}\NormalTok{, vorname, nachname, }\AttributeTok{sep =} \StringTok{" "}\NormalTok{)}
\end{Highlighting}
\end{Shaded}

\begin{verbatim}
## [1] "Mein Name ist: Ronald Aylmer Fisher"
\end{verbatim}

\hypertarget{logical-vectors}{%
\subsubsection{Logical vectors}\label{logical-vectors}}

Logische Vektoren werden vor allem dazu benutzt, um numerische Vektoren
zu indizieren, z.B. um alle positiven Elemente eines Vektors
auszuwählen.

\begin{Shaded}
\begin{Highlighting}[]
\NormalTok{x }\OtherTok{\textless{}{-}} \FunctionTok{rnorm}\NormalTok{(}\DecValTok{24}\NormalTok{)}
\NormalTok{x}
\end{Highlighting}
\end{Shaded}

\begin{verbatim}
##  [1]  0.67719270  0.31045298 -0.51651979  0.68036338  0.56341661  1.03551009
##  [7]  0.08964888 -0.13865505 -0.69764687 -1.91330317  1.80536838  0.36199042
## [13] -1.08775222 -0.07500418  0.56021014  1.76708829  0.12882955  0.41319518
## [19]  0.86110875  1.73572936  0.27503446 -0.64168612 -0.03523950  0.64335096
\end{verbatim}

\begin{Shaded}
\begin{Highlighting}[]
\NormalTok{x }\SpecialCharTok{\textgreater{}} \DecValTok{0}
\end{Highlighting}
\end{Shaded}

\begin{verbatim}
##  [1]  TRUE  TRUE FALSE  TRUE  TRUE  TRUE  TRUE FALSE FALSE FALSE  TRUE  TRUE
## [13] FALSE FALSE  TRUE  TRUE  TRUE  TRUE  TRUE  TRUE  TRUE FALSE FALSE  TRUE
\end{verbatim}

\begin{Shaded}
\begin{Highlighting}[]
\FunctionTok{mean}\NormalTok{(x)}
\end{Highlighting}
\end{Shaded}

\begin{verbatim}
## [1] 0.2834451
\end{verbatim}

man kann auch bei dieses Vektors den Mittelwert bestimmen

\hypertarget{factors}{%
\subsubsection{Factors}\label{factors}}

\begin{Shaded}
\begin{Highlighting}[]
\NormalTok{geschlecht }\OtherTok{\textless{}{-}} \FunctionTok{c}\NormalTok{(}\StringTok{"male"}\NormalTok{, }\StringTok{"female"}\NormalTok{, }\StringTok{"male"}\NormalTok{, }\StringTok{"male"}\NormalTok{, }\StringTok{"female"}\NormalTok{)}
\NormalTok{geschlecht}
\end{Highlighting}
\end{Shaded}

\begin{verbatim}
## [1] "male"   "female" "male"   "male"   "female"
\end{verbatim}

\begin{Shaded}
\begin{Highlighting}[]
\FunctionTok{typeof}\NormalTok{(geschlecht)}
\end{Highlighting}
\end{Shaded}

\begin{verbatim}
## [1] "character"
\end{verbatim}

\begin{Shaded}
\begin{Highlighting}[]
\NormalTok{geschlecht }\OtherTok{\textless{}{-}} \FunctionTok{factor}\NormalTok{(geschlecht, }\AttributeTok{levels =} \FunctionTok{c}\NormalTok{(}\StringTok{"female"}\NormalTok{, }\StringTok{"male"}\NormalTok{))}
\NormalTok{geschlecht}
\end{Highlighting}
\end{Shaded}

\begin{verbatim}
## [1] male   female male   male   female
## Levels: female male
\end{verbatim}

\begin{Shaded}
\begin{Highlighting}[]
\FunctionTok{typeof}\NormalTok{(geschlecht)}
\end{Highlighting}
\end{Shaded}

\begin{verbatim}
## [1] "integer"
\end{verbatim}

Hier kann man sehen das durch das Anwenden der methode Faktor man
verschhiedene level bestimmen kann --\textgreater{} so werden die Werte
zu Integern und geschlecht zu der Klasse Faktor Mit factor() kann man
die Reihenfolge genau bestimmen - es müssen aber alle Stufen explizit
angegeben werden.

\hypertarget{lists}{%
\subsubsection{Lists}\label{lists}}

\begin{Shaded}
\begin{Highlighting}[]
\NormalTok{x }\OtherTok{\textless{}{-}} \FunctionTok{list}\NormalTok{(}\DecValTok{1}\SpecialCharTok{:}\DecValTok{3}\NormalTok{, }\StringTok{"a"}\NormalTok{, }\FunctionTok{c}\NormalTok{(}\ConstantTok{TRUE}\NormalTok{, }\ConstantTok{FALSE}\NormalTok{, }\ConstantTok{TRUE}\NormalTok{), }\FunctionTok{c}\NormalTok{(}\FloatTok{2.3}\NormalTok{, }\FloatTok{5.9}\NormalTok{))}
\NormalTok{x}
\end{Highlighting}
\end{Shaded}

\begin{verbatim}
## [[1]]
## [1] 1 2 3
## 
## [[2]]
## [1] "a"
## 
## [[3]]
## [1]  TRUE FALSE  TRUE
## 
## [[4]]
## [1] 2.3 5.9
\end{verbatim}

\begin{Shaded}
\begin{Highlighting}[]
\NormalTok{x[}\DecValTok{2}\NormalTok{]}
\end{Highlighting}
\end{Shaded}

\begin{verbatim}
## [[1]]
## [1] "a"
\end{verbatim}

\begin{Shaded}
\begin{Highlighting}[]
\NormalTok{b }\OtherTok{\textless{}{-}} \FunctionTok{list}\NormalTok{(}\AttributeTok{int =} \DecValTok{1}\SpecialCharTok{:}\DecValTok{3}\NormalTok{,}
          \AttributeTok{string =} \StringTok{"a"}\NormalTok{,}
          \AttributeTok{log =} \FunctionTok{c}\NormalTok{(}\ConstantTok{TRUE}\NormalTok{, }\ConstantTok{FALSE}\NormalTok{, }\ConstantTok{TRUE}\NormalTok{),}
          \AttributeTok{double =} \FunctionTok{c}\NormalTok{(}\FloatTok{2.3}\NormalTok{, }\FloatTok{5.9}\NormalTok{))}
\NormalTok{b}
\end{Highlighting}
\end{Shaded}

\begin{verbatim}
## $int
## [1] 1 2 3
## 
## $string
## [1] "a"
## 
## $log
## [1]  TRUE FALSE  TRUE
## 
## $double
## [1] 2.3 5.9
\end{verbatim}

\hypertarget{data-frames}{%
\subsubsection{Data Frames}\label{data-frames}}

Nun kommen wir zu dem für uns wichtigsten Objekt in R, dem Data Frame.
Datensätze werden in R durch Data Frames repräsentiert. Ein Data Frame
besteht aus Zeilen (rows) und Spalten (columns). Technisch gesehen ist
ein Data Frame eine Liste, deren Elemente gleichlange (equal-length)
Vektoren sind.

\begin{Shaded}
\begin{Highlighting}[]
\FunctionTok{library}\NormalTok{(dplyr)}
\end{Highlighting}
\end{Shaded}

\begin{verbatim}
## 
## Attache Paket: 'dplyr'
\end{verbatim}

\begin{verbatim}
## Die folgenden Objekte sind maskiert von 'package:stats':
## 
##     filter, lag
\end{verbatim}

\begin{verbatim}
## Die folgenden Objekte sind maskiert von 'package:base':
## 
##     intersect, setdiff, setequal, union
\end{verbatim}

\begin{Shaded}
\begin{Highlighting}[]
\NormalTok{df }\OtherTok{\textless{}{-}} \FunctionTok{tibble}\NormalTok{(}\AttributeTok{geschlecht =} \FunctionTok{factor}\NormalTok{(}\FunctionTok{c}\NormalTok{(}\StringTok{"male"}\NormalTok{, }\StringTok{"female"}\NormalTok{,}
                                   \StringTok{"male"}\NormalTok{, }\StringTok{"male"}\NormalTok{,}
                                   \StringTok{"female"}\NormalTok{)),}
             \AttributeTok{alter =} \FunctionTok{c}\NormalTok{(}\DecValTok{22}\NormalTok{, }\DecValTok{45}\NormalTok{, }\DecValTok{33}\NormalTok{, }\DecValTok{27}\NormalTok{, }\DecValTok{30}\NormalTok{))}
\NormalTok{df}
\end{Highlighting}
\end{Shaded}

\begin{verbatim}
## # A tibble: 5 x 2
##   geschlecht alter
##   <fct>      <dbl>
## 1 male          22
## 2 female        45
## 3 male          33
## 4 male          27
## 5 female        30
\end{verbatim}

\begin{Shaded}
\begin{Highlighting}[]
\FunctionTok{attributes}\NormalTok{(df)}
\end{Highlighting}
\end{Shaded}

\begin{verbatim}
## $class
## [1] "tbl_df"     "tbl"        "data.frame"
## 
## $row.names
## [1] 1 2 3 4 5
## 
## $names
## [1] "geschlecht" "alter"
\end{verbatim}

\hypertarget{w3schools}{%
\subsection{W3schools:}\label{w3schools}}

\hypertarget{r-tutorial}{%
\subsubsection{R Tutorial:}\label{r-tutorial}}

\hypertarget{r-comments}{%
\paragraph{R Comments:}\label{r-comments}}

\begin{Shaded}
\begin{Highlighting}[]
\CommentTok{\# This is a comment}
\StringTok{"Hello World!"}
\end{Highlighting}
\end{Shaded}

\begin{verbatim}
## [1] "Hello World!"
\end{verbatim}

\hypertarget{r-variables}{%
\paragraph{R Variables:}\label{r-variables}}

\begin{Shaded}
\begin{Highlighting}[]
\NormalTok{name }\OtherTok{\textless{}{-}} \StringTok{"John"}
\NormalTok{age }\OtherTok{\textless{}{-}} \DecValTok{40}

\NormalTok{name   }\CommentTok{\# output "John"}
\end{Highlighting}
\end{Shaded}

\begin{verbatim}
## [1] "John"
\end{verbatim}

\begin{Shaded}
\begin{Highlighting}[]
\NormalTok{age    }\CommentTok{\# output 40}
\end{Highlighting}
\end{Shaded}

\begin{verbatim}
## [1] 40
\end{verbatim}

\begin{Shaded}
\begin{Highlighting}[]
\CommentTok{\# Concatenate Elements:}
\NormalTok{text }\OtherTok{\textless{}{-}} \StringTok{"awesome"}

\FunctionTok{paste}\NormalTok{(}\StringTok{"R is"}\NormalTok{, text)}
\end{Highlighting}
\end{Shaded}

\begin{verbatim}
## [1] "R is awesome"
\end{verbatim}

\begin{Shaded}
\begin{Highlighting}[]
\CommentTok{\# Multiple Variables:}

\CommentTok{\# Assign the same value to multiple variables in one line:}
\NormalTok{var1 }\OtherTok{\textless{}{-}}\NormalTok{ var2 }\OtherTok{\textless{}{-}}\NormalTok{ var3 }\OtherTok{\textless{}{-}} \StringTok{"Orange"}

\CommentTok{\# Print variable values:}
\NormalTok{var1}
\end{Highlighting}
\end{Shaded}

\begin{verbatim}
## [1] "Orange"
\end{verbatim}

\begin{Shaded}
\begin{Highlighting}[]
\NormalTok{var2}
\end{Highlighting}
\end{Shaded}

\begin{verbatim}
## [1] "Orange"
\end{verbatim}

\begin{Shaded}
\begin{Highlighting}[]
\NormalTok{var3}
\end{Highlighting}
\end{Shaded}

\begin{verbatim}
## [1] "Orange"
\end{verbatim}

\begin{Shaded}
\begin{Highlighting}[]
\CommentTok{\# Variabel Names:}

\CommentTok{\# Legal variable names:}
\NormalTok{myvar }\OtherTok{\textless{}{-}} \StringTok{"John"}
\NormalTok{my\_var }\OtherTok{\textless{}{-}} \StringTok{"John"}
\NormalTok{myVar }\OtherTok{\textless{}{-}} \StringTok{"John"}
\NormalTok{MYVAR }\OtherTok{\textless{}{-}} \StringTok{"John"}
\NormalTok{myvar2 }\OtherTok{\textless{}{-}} \StringTok{"John"}
\NormalTok{.myvar }\OtherTok{\textless{}{-}} \StringTok{"John"}

\CommentTok{\# Illegal variable names:}
\CommentTok{\# 2myvar \textless{}{-} "John"}
\CommentTok{\# my{-}var \textless{}{-} "John"}
\CommentTok{\# my var \textless{}{-} "John"}
\CommentTok{\# \_my\_var \textless{}{-} "John"}
\CommentTok{\# my\_v@ar \textless{}{-} "John"}
\CommentTok{\# TRUE \textless{}{-} "John"}
\end{Highlighting}
\end{Shaded}

\hypertarget{r-data-types}{%
\paragraph{R Data Types:}\label{r-data-types}}

\begin{enumerate}
\def\labelenumi{\arabic{enumi}.}
\tightlist
\item
  numeric - (10.5, 55, 787)
\item
  integer - (1L, 55L, 100L, where the letter ``L'' declares this as an
  integer)
\item
  complex - (9 + 3i, where ``i'' is the imaginary part)
\item
  character (a.k.a. string) - (``k'', ``R is exciting'', ``FALSE'',
  ``11.5'')
\item
  logical (a.k.a. boolean) - (TRUE or FALSE)
\end{enumerate}

\begin{Shaded}
\begin{Highlighting}[]
\CommentTok{\# numeric}
\NormalTok{x }\OtherTok{\textless{}{-}} \FloatTok{10.5}
\FunctionTok{class}\NormalTok{(x)}
\end{Highlighting}
\end{Shaded}

\begin{verbatim}
## [1] "numeric"
\end{verbatim}

\begin{Shaded}
\begin{Highlighting}[]
\CommentTok{\# integer}
\NormalTok{x }\OtherTok{\textless{}{-}}\NormalTok{ 1000L}
\FunctionTok{class}\NormalTok{(x)}
\end{Highlighting}
\end{Shaded}

\begin{verbatim}
## [1] "integer"
\end{verbatim}

\begin{Shaded}
\begin{Highlighting}[]
\CommentTok{\# complex}
\NormalTok{x }\OtherTok{\textless{}{-}}\NormalTok{ 9i }\SpecialCharTok{+} \DecValTok{3}
\FunctionTok{class}\NormalTok{(x)}
\end{Highlighting}
\end{Shaded}

\begin{verbatim}
## [1] "complex"
\end{verbatim}

\begin{Shaded}
\begin{Highlighting}[]
\CommentTok{\# character/string}
\NormalTok{x }\OtherTok{\textless{}{-}} \StringTok{"R is exciting"}
\FunctionTok{class}\NormalTok{(x)}
\end{Highlighting}
\end{Shaded}

\begin{verbatim}
## [1] "character"
\end{verbatim}

\begin{Shaded}
\begin{Highlighting}[]
\CommentTok{\# logical/boolean}
\NormalTok{x }\OtherTok{\textless{}{-}} \ConstantTok{TRUE}
\FunctionTok{class}\NormalTok{(x)}
\end{Highlighting}
\end{Shaded}

\begin{verbatim}
## [1] "logical"
\end{verbatim}

\hypertarget{r-numbers}{%
\paragraph{R Numbers:}\label{r-numbers}}

\begin{Shaded}
\begin{Highlighting}[]
\CommentTok{\# R Numbers:}
\NormalTok{x }\OtherTok{\textless{}{-}} \FloatTok{10.5}   \CommentTok{\# numeric}
\NormalTok{y }\OtherTok{\textless{}{-}}\NormalTok{ 10L    }\CommentTok{\# integer}
\NormalTok{z }\OtherTok{\textless{}{-}}\NormalTok{ 1i     }\CommentTok{\# complex}

\CommentTok{\# Type Conversion:}
\FunctionTok{as.numeric}\NormalTok{()}
\end{Highlighting}
\end{Shaded}

\begin{verbatim}
## numeric(0)
\end{verbatim}

\begin{Shaded}
\begin{Highlighting}[]
\FunctionTok{as.integer}\NormalTok{()}
\end{Highlighting}
\end{Shaded}

\begin{verbatim}
## integer(0)
\end{verbatim}

\begin{Shaded}
\begin{Highlighting}[]
\FunctionTok{as.complex}\NormalTok{()}
\end{Highlighting}
\end{Shaded}

\begin{verbatim}
## complex(0)
\end{verbatim}

\hypertarget{r-math}{%
\paragraph{R Math:}\label{r-math}}

\begin{Shaded}
\begin{Highlighting}[]
\FunctionTok{max}\NormalTok{(}\DecValTok{5}\NormalTok{, }\DecValTok{10}\NormalTok{, }\DecValTok{15}\NormalTok{)}
\end{Highlighting}
\end{Shaded}

\begin{verbatim}
## [1] 15
\end{verbatim}

\begin{Shaded}
\begin{Highlighting}[]
\FunctionTok{min}\NormalTok{(}\DecValTok{5}\NormalTok{, }\DecValTok{10}\NormalTok{, }\DecValTok{15}\NormalTok{)}
\end{Highlighting}
\end{Shaded}

\begin{verbatim}
## [1] 5
\end{verbatim}

\begin{Shaded}
\begin{Highlighting}[]
\FunctionTok{sqrt}\NormalTok{(}\DecValTok{16}\NormalTok{)}
\end{Highlighting}
\end{Shaded}

\begin{verbatim}
## [1] 4
\end{verbatim}

\begin{Shaded}
\begin{Highlighting}[]
\FunctionTok{abs}\NormalTok{(}\SpecialCharTok{{-}}\FloatTok{4.7}\NormalTok{) }\CommentTok{\# Returns positive Value}
\end{Highlighting}
\end{Shaded}

\begin{verbatim}
## [1] 4.7
\end{verbatim}

\begin{Shaded}
\begin{Highlighting}[]
\FunctionTok{ceiling}\NormalTok{(}\FloatTok{1.4}\NormalTok{) }\CommentTok{\# Rounds the number up}
\end{Highlighting}
\end{Shaded}

\begin{verbatim}
## [1] 2
\end{verbatim}

\begin{Shaded}
\begin{Highlighting}[]
\FunctionTok{floor}\NormalTok{(}\FloatTok{1.4}\NormalTok{) }\CommentTok{\# Rounds the number down}
\end{Highlighting}
\end{Shaded}

\begin{verbatim}
## [1] 1
\end{verbatim}

\hypertarget{r-strings}{%
\paragraph{R Strings:}\label{r-strings}}

\begin{Shaded}
\begin{Highlighting}[]
\NormalTok{str }\OtherTok{\textless{}{-}} \StringTok{"Hello"}
\NormalTok{str }\CommentTok{\# print the value of str}
\end{Highlighting}
\end{Shaded}

\begin{verbatim}
## [1] "Hello"
\end{verbatim}

\begin{Shaded}
\begin{Highlighting}[]
\CommentTok{\# For multiple lines use ,}
\NormalTok{str }\OtherTok{\textless{}{-}} \StringTok{"Lorem ipsum dolor sit amet,}
\StringTok{consectetur adipiscing elit,}
\StringTok{sed do eiusmod tempor incididunt}
\StringTok{ut labore et dolore magna aliqua."}

\NormalTok{str }\CommentTok{\# print the value of str}
\end{Highlighting}
\end{Shaded}

\begin{verbatim}
## [1] "Lorem ipsum dolor sit amet,\nconsectetur adipiscing elit,\nsed do eiusmod tempor incididunt\nut labore et dolore magna aliqua."
\end{verbatim}

\begin{Shaded}
\begin{Highlighting}[]
\FunctionTok{cat}\NormalTok{(str) }\CommentTok{\# To set the line brakes at the same position as the code}
\end{Highlighting}
\end{Shaded}

\begin{verbatim}
## Lorem ipsum dolor sit amet,
## consectetur adipiscing elit,
## sed do eiusmod tempor incididunt
## ut labore et dolore magna aliqua.
\end{verbatim}

\begin{Shaded}
\begin{Highlighting}[]
\FunctionTok{nchar}\NormalTok{(str) }\CommentTok{\# To get the length of a string}
\end{Highlighting}
\end{Shaded}

\begin{verbatim}
## [1] 123
\end{verbatim}

\begin{Shaded}
\begin{Highlighting}[]
\CommentTok{\# For finding characters and sequences}
\NormalTok{str }\OtherTok{\textless{}{-}} \StringTok{"Hello World!"}

\FunctionTok{grepl}\NormalTok{(}\StringTok{"H"}\NormalTok{, str)}
\end{Highlighting}
\end{Shaded}

\begin{verbatim}
## [1] TRUE
\end{verbatim}

\begin{Shaded}
\begin{Highlighting}[]
\FunctionTok{grepl}\NormalTok{(}\StringTok{"Hello"}\NormalTok{, str)}
\end{Highlighting}
\end{Shaded}

\begin{verbatim}
## [1] TRUE
\end{verbatim}

\begin{Shaded}
\begin{Highlighting}[]
\FunctionTok{grepl}\NormalTok{(}\StringTok{"X"}\NormalTok{, str)}
\end{Highlighting}
\end{Shaded}

\begin{verbatim}
## [1] FALSE
\end{verbatim}

Escape Symbols:

\begin{enumerate}
\def\labelenumi{\arabic{enumi}.}
\tightlist
\item
  \textbackslash{} Backslash
\item
  \textbackslash{} New Line
\item
  \r   Carriage Return
\item
  \t   Tab
\item
  \b   Backspace
\end{enumerate}

\hypertarget{r-operators}{%
\paragraph{R Operators:}\label{r-operators}}

Arithmetic Operators:

\begin{enumerate}
\def\labelenumi{\arabic{enumi}.}
\item
  \begin{itemize}
  \tightlist
  \item
    Addition
  \end{itemize}
\item
  \begin{itemize}
  \tightlist
  \item
    Subtraction
  \end{itemize}
\item
  \begin{itemize}
  \tightlist
  \item
    Multiplication\\
  \end{itemize}
\item
  / Division\\
\item
  \texttt{\^{}} Exponent\\
\item
  \%\% Modulus (Remainder from division)\\
\item
  \%/\% Integer Division
\end{enumerate}

Assignment Operators:

\begin{Shaded}
\begin{Highlighting}[]
\NormalTok{my\_var }\OtherTok{\textless{}{-}} \DecValTok{3}

\NormalTok{my\_var }\OtherTok{\textless{}\textless{}{-}} \DecValTok{3}

\DecValTok{3} \OtherTok{{-}\textgreater{}}\NormalTok{ my\_var}

\DecValTok{3} \OtherTok{{-}\textgreater{}\textgreater{}}\NormalTok{ my\_var}

\NormalTok{my\_var }\CommentTok{\# print my\_var}
\end{Highlighting}
\end{Shaded}

\begin{verbatim}
## [1] 3
\end{verbatim}

Comparison Operators:

\begin{enumerate}
\def\labelenumi{\arabic{enumi}.}
\item
  == Equal\\
\item
  != Not equal\\
\item
  \begin{quote}
  Greater than
  \end{quote}
\item
  \textless{} Less than\\
\item
  \begin{quote}
  = Greater than or equal to
  \end{quote}
\item
  \textless= Less than or equal to
\end{enumerate}

Logical Operators:

\begin{enumerate}
\def\labelenumi{\arabic{enumi}.}
\item
  \& Element-wise Logical AND operator. It returns TRUE if both elements
  are TRUE
\item
  \&\& Logical AND operator - Returns TRUE if both statements are TRUE
\item
  ~~~Elementwise- Logical OR operator. It returns TRUE if one of the
  statement is TRUE
\item
  \textbar\textbar{} Logical OR operator. It returns TRUE if one of the
  statement is TRUE.
\item
  ! Logical NOT - returns FALSE if statement is TRUE
\end{enumerate}

Miscellaneous Operators:

\begin{enumerate}
\def\labelenumi{\arabic{enumi}.}
\tightlist
\item
  : Creates a series of numbers in a sequence\\
\item
  \%in\% Find out if an element belongs to a vector\\
\item
  \%*\% Matrix Multiplication
\end{enumerate}

\hypertarget{r-functions}{%
\paragraph{R Functions:}\label{r-functions}}

\begin{Shaded}
\begin{Highlighting}[]
\NormalTok{str }\OtherTok{\textless{}{-}} \StringTok{"Hello"}
\NormalTok{str }\CommentTok{\# print the value of str}
\end{Highlighting}
\end{Shaded}

\begin{verbatim}
## [1] "Hello"
\end{verbatim}

\begin{Shaded}
\begin{Highlighting}[]
\CommentTok{\# For multiple lines use ,}
\NormalTok{str }\OtherTok{\textless{}{-}} \StringTok{"Lorem ipsum dolor sit amet,}
\StringTok{consectetur adipiscing elit,}
\StringTok{sed do eiusmod tempor incididunt}
\StringTok{ut labore et dolore magna aliqua."}

\NormalTok{str }\CommentTok{\# print the value of str}
\end{Highlighting}
\end{Shaded}

\begin{verbatim}
## [1] "Lorem ipsum dolor sit amet,\nconsectetur adipiscing elit,\nsed do eiusmod tempor incididunt\nut labore et dolore magna aliqua."
\end{verbatim}

\begin{Shaded}
\begin{Highlighting}[]
\FunctionTok{cat}\NormalTok{(str) }\CommentTok{\# To set the line brakes at the same position as the code}
\end{Highlighting}
\end{Shaded}

\begin{verbatim}
## Lorem ipsum dolor sit amet,
## consectetur adipiscing elit,
## sed do eiusmod tempor incididunt
## ut labore et dolore magna aliqua.
\end{verbatim}

\begin{Shaded}
\begin{Highlighting}[]
\FunctionTok{nchar}\NormalTok{(str) }\CommentTok{\# To get the length of a string}
\end{Highlighting}
\end{Shaded}

\begin{verbatim}
## [1] 123
\end{verbatim}

\begin{Shaded}
\begin{Highlighting}[]
\CommentTok{\# For finding characters and sequences}
\NormalTok{str }\OtherTok{\textless{}{-}} \StringTok{"Hello World!"}

\FunctionTok{grepl}\NormalTok{(}\StringTok{"H"}\NormalTok{, str)}
\end{Highlighting}
\end{Shaded}

\begin{verbatim}
## [1] TRUE
\end{verbatim}

\begin{Shaded}
\begin{Highlighting}[]
\FunctionTok{grepl}\NormalTok{(}\StringTok{"Hello"}\NormalTok{, str)}
\end{Highlighting}
\end{Shaded}

\begin{verbatim}
## [1] TRUE
\end{verbatim}

\begin{Shaded}
\begin{Highlighting}[]
\FunctionTok{grepl}\NormalTok{(}\StringTok{"X"}\NormalTok{, str)}
\end{Highlighting}
\end{Shaded}

\begin{verbatim}
## [1] FALSE
\end{verbatim}

Escape Symbols:

\begin{enumerate}
\def\labelenumi{\arabic{enumi}.}
\tightlist
\item
  \textbackslash{} Backslash
\item
  \textbackslash{} New Line
\item
  \r   Carriage Return
\item
  \t   Tab
\item
  \b   Backspace
\end{enumerate}

\hypertarget{r-operators-1}{%
\paragraph{R Operators:}\label{r-operators-1}}

Arithmetic Operators:

\begin{enumerate}
\def\labelenumi{\arabic{enumi}.}
\item
  \begin{itemize}
  \tightlist
  \item
    Addition
  \end{itemize}
\item
  \begin{itemize}
  \tightlist
  \item
    Subtraction
  \end{itemize}
\item
  \begin{itemize}
  \tightlist
  \item
    Multiplication\\
  \end{itemize}
\item
  / Division\\
\item
  \^{} Exponent\\
\item
  \%\% Modulus (Remainder from division)\\
\item
  \%/\% Integer Division
\end{enumerate}

Assignment Operators:

\begin{Shaded}
\begin{Highlighting}[]
\NormalTok{my\_function }\OtherTok{\textless{}{-}} \ControlFlowTok{function}\NormalTok{() \{}
  \FunctionTok{print}\NormalTok{(}\StringTok{"Hello World!"}\NormalTok{)}
\NormalTok{\}}

\FunctionTok{my\_function}\NormalTok{() }\CommentTok{\# call the function named my\_function}
\end{Highlighting}
\end{Shaded}

\begin{verbatim}
## [1] "Hello World!"
\end{verbatim}

\hypertarget{r-data-structures}{%
\subsubsection{R Data Structures:}\label{r-data-structures}}

\hypertarget{r-vectors}{%
\paragraph{R Vectors:}\label{r-vectors}}

\begin{Shaded}
\begin{Highlighting}[]
\CommentTok{\# Vector of strings}
\NormalTok{fruits }\OtherTok{\textless{}{-}} \FunctionTok{c}\NormalTok{(}\StringTok{"banana"}\NormalTok{, }\StringTok{"apple"}\NormalTok{, }\StringTok{"orange"}\NormalTok{)}

\CommentTok{\# Print fruits}
\NormalTok{fruits}
\end{Highlighting}
\end{Shaded}

\begin{verbatim}
## [1] "banana" "apple"  "orange"
\end{verbatim}

\begin{Shaded}
\begin{Highlighting}[]
\CommentTok{\# Vector of numerical values}
\NormalTok{numbers }\OtherTok{\textless{}{-}} \FunctionTok{c}\NormalTok{(}\DecValTok{1}\NormalTok{, }\DecValTok{2}\NormalTok{, }\DecValTok{3}\NormalTok{)}

\CommentTok{\# Print numbers}
\NormalTok{numbers}
\end{Highlighting}
\end{Shaded}

\begin{verbatim}
## [1] 1 2 3
\end{verbatim}

\begin{Shaded}
\begin{Highlighting}[]
\CommentTok{\# Vector with numerical values in a sequence}
\NormalTok{numbers }\OtherTok{\textless{}{-}} \DecValTok{1}\SpecialCharTok{:}\DecValTok{10}

\NormalTok{numbers}
\end{Highlighting}
\end{Shaded}

\begin{verbatim}
##  [1]  1  2  3  4  5  6  7  8  9 10
\end{verbatim}

\begin{Shaded}
\begin{Highlighting}[]
\CommentTok{\# Length of vectors:}
\NormalTok{fruits }\OtherTok{\textless{}{-}} \FunctionTok{c}\NormalTok{(}\StringTok{"banana"}\NormalTok{, }\StringTok{"apple"}\NormalTok{, }\StringTok{"orange"}\NormalTok{)}

\FunctionTok{length}\NormalTok{(fruits)}
\end{Highlighting}
\end{Shaded}

\begin{verbatim}
## [1] 3
\end{verbatim}

\begin{Shaded}
\begin{Highlighting}[]
\CommentTok{\# Sort a vector:}
\NormalTok{fruits }\OtherTok{\textless{}{-}} \FunctionTok{c}\NormalTok{(}\StringTok{"banana"}\NormalTok{, }\StringTok{"apple"}\NormalTok{, }\StringTok{"orange"}\NormalTok{, }\StringTok{"mango"}\NormalTok{, }\StringTok{"lemon"}\NormalTok{)}
\NormalTok{numbers }\OtherTok{\textless{}{-}} \FunctionTok{c}\NormalTok{(}\DecValTok{13}\NormalTok{, }\DecValTok{3}\NormalTok{, }\DecValTok{5}\NormalTok{, }\DecValTok{7}\NormalTok{, }\DecValTok{20}\NormalTok{, }\DecValTok{2}\NormalTok{)}

\FunctionTok{sort}\NormalTok{(fruits)  }\CommentTok{\# Sort a string}
\end{Highlighting}
\end{Shaded}

\begin{verbatim}
## [1] "apple"  "banana" "lemon"  "mango"  "orange"
\end{verbatim}

\begin{Shaded}
\begin{Highlighting}[]
\FunctionTok{sort}\NormalTok{(numbers) }\CommentTok{\# Sort numbers}
\end{Highlighting}
\end{Shaded}

\begin{verbatim}
## [1]  2  3  5  7 13 20
\end{verbatim}

\begin{Shaded}
\begin{Highlighting}[]
\CommentTok{\# Accessing vectors:}
\NormalTok{fruits[}\DecValTok{1}\NormalTok{]}
\end{Highlighting}
\end{Shaded}

\begin{verbatim}
## [1] "banana"
\end{verbatim}

\begin{Shaded}
\begin{Highlighting}[]
\CommentTok{\# Change "banana" to "pear"}
\NormalTok{fruits[}\DecValTok{1}\NormalTok{] }\OtherTok{\textless{}{-}} \StringTok{"pear"}

\CommentTok{\# Print fruits}
\NormalTok{fruits}
\end{Highlighting}
\end{Shaded}

\begin{verbatim}
## [1] "pear"   "apple"  "orange" "mango"  "lemon"
\end{verbatim}

\begin{Shaded}
\begin{Highlighting}[]
\CommentTok{\# Repeat vectors:}
\NormalTok{repeat\_each }\OtherTok{\textless{}{-}} \FunctionTok{rep}\NormalTok{(}\FunctionTok{c}\NormalTok{(}\DecValTok{1}\NormalTok{,}\DecValTok{2}\NormalTok{,}\DecValTok{3}\NormalTok{), }\AttributeTok{each =} \DecValTok{3}\NormalTok{)}

\NormalTok{repeat\_each}
\end{Highlighting}
\end{Shaded}

\begin{verbatim}
## [1] 1 1 1 2 2 2 3 3 3
\end{verbatim}

\begin{Shaded}
\begin{Highlighting}[]
\NormalTok{numbers }\OtherTok{\textless{}{-}} \FunctionTok{seq}\NormalTok{(}\AttributeTok{from =} \DecValTok{0}\NormalTok{, }\AttributeTok{to =} \DecValTok{100}\NormalTok{, }\AttributeTok{by =} \DecValTok{20}\NormalTok{)}

\NormalTok{numbers}
\end{Highlighting}
\end{Shaded}

\begin{verbatim}
## [1]   0  20  40  60  80 100
\end{verbatim}

\hypertarget{r-lists}{%
\paragraph{R Lists:}\label{r-lists}}

\begin{Shaded}
\begin{Highlighting}[]
\CommentTok{\# List of strings}
\NormalTok{thislist }\OtherTok{\textless{}{-}} \FunctionTok{list}\NormalTok{(}\StringTok{"apple"}\NormalTok{, }\StringTok{"banana"}\NormalTok{, }\StringTok{"cherry"}\NormalTok{)}

\CommentTok{\# Print the list}
\NormalTok{thislist}
\end{Highlighting}
\end{Shaded}

\begin{verbatim}
## [[1]]
## [1] "apple"
## 
## [[2]]
## [1] "banana"
## 
## [[3]]
## [1] "cherry"
\end{verbatim}

\begin{Shaded}
\begin{Highlighting}[]
\CommentTok{\# Access the list}
\NormalTok{thislist[}\DecValTok{1}\NormalTok{]}
\end{Highlighting}
\end{Shaded}

\begin{verbatim}
## [[1]]
## [1] "apple"
\end{verbatim}

\begin{Shaded}
\begin{Highlighting}[]
\CommentTok{\# Change item in list}
\NormalTok{thislist }\OtherTok{\textless{}{-}} \FunctionTok{list}\NormalTok{(}\StringTok{"apple"}\NormalTok{, }\StringTok{"banana"}\NormalTok{, }\StringTok{"cherry"}\NormalTok{)}
\NormalTok{thislist[}\DecValTok{1}\NormalTok{] }\OtherTok{\textless{}{-}} \StringTok{"blackcurrant"}

\CommentTok{\# Length of a list}
\FunctionTok{length}\NormalTok{(thislist)}
\end{Highlighting}
\end{Shaded}

\begin{verbatim}
## [1] 3
\end{verbatim}

\begin{Shaded}
\begin{Highlighting}[]
\CommentTok{\# Check if item exists}
\StringTok{"apple"} \SpecialCharTok{\%in\%}\NormalTok{ thislist}
\end{Highlighting}
\end{Shaded}

\begin{verbatim}
## [1] FALSE
\end{verbatim}

\begin{Shaded}
\begin{Highlighting}[]
\CommentTok{\# Add item to list}
\FunctionTok{append}\NormalTok{(thislist, }\StringTok{"orange"}\NormalTok{)}
\end{Highlighting}
\end{Shaded}

\begin{verbatim}
## [[1]]
## [1] "blackcurrant"
## 
## [[2]]
## [1] "banana"
## 
## [[3]]
## [1] "cherry"
## 
## [[4]]
## [1] "orange"
\end{verbatim}

\begin{Shaded}
\begin{Highlighting}[]
\CommentTok{\# Remove item from list}
\NormalTok{newlist }\OtherTok{\textless{}{-}}\NormalTok{ thislist[}\SpecialCharTok{{-}}\DecValTok{1}\NormalTok{]}

\CommentTok{\# Range of indexes}
\NormalTok{(thislist)[}\DecValTok{2}\SpecialCharTok{:}\DecValTok{5}\NormalTok{]}
\end{Highlighting}
\end{Shaded}

\begin{verbatim}
## [[1]]
## [1] "banana"
## 
## [[2]]
## [1] "cherry"
## 
## [[3]]
## NULL
## 
## [[4]]
## NULL
\end{verbatim}

\begin{Shaded}
\begin{Highlighting}[]
\CommentTok{\# Loop through list}
\ControlFlowTok{for}\NormalTok{ (x }\ControlFlowTok{in}\NormalTok{ thislist) \{}
  \FunctionTok{print}\NormalTok{(x)}
\NormalTok{\}}
\end{Highlighting}
\end{Shaded}

\begin{verbatim}
## [1] "blackcurrant"
## [1] "banana"
## [1] "cherry"
\end{verbatim}

\begin{Shaded}
\begin{Highlighting}[]
\CommentTok{\# Join two lists}
\NormalTok{list1 }\OtherTok{\textless{}{-}} \FunctionTok{list}\NormalTok{(}\StringTok{"a"}\NormalTok{, }\StringTok{"b"}\NormalTok{, }\StringTok{"c"}\NormalTok{)}
\NormalTok{list2 }\OtherTok{\textless{}{-}} \FunctionTok{list}\NormalTok{(}\DecValTok{1}\NormalTok{,}\DecValTok{2}\NormalTok{,}\DecValTok{3}\NormalTok{)}
\NormalTok{list3 }\OtherTok{\textless{}{-}} \FunctionTok{c}\NormalTok{(list1,list2)}

\NormalTok{list3}
\end{Highlighting}
\end{Shaded}

\begin{verbatim}
## [[1]]
## [1] "a"
## 
## [[2]]
## [1] "b"
## 
## [[3]]
## [1] "c"
## 
## [[4]]
## [1] 1
## 
## [[5]]
## [1] 2
## 
## [[6]]
## [1] 3
\end{verbatim}

\hypertarget{r-matrices}{%
\paragraph{R Matrices:}\label{r-matrices}}

\begin{Shaded}
\begin{Highlighting}[]
\CommentTok{\# Create a matrix}
\NormalTok{thismatrix }\OtherTok{\textless{}{-}} \FunctionTok{matrix}\NormalTok{(}\FunctionTok{c}\NormalTok{(}\DecValTok{1}\NormalTok{,}\DecValTok{2}\NormalTok{,}\DecValTok{3}\NormalTok{,}\DecValTok{4}\NormalTok{,}\DecValTok{5}\NormalTok{,}\DecValTok{6}\NormalTok{), }\AttributeTok{nrow =} \DecValTok{3}\NormalTok{, }\AttributeTok{ncol =} \DecValTok{2}\NormalTok{)}

\CommentTok{\# Print the matrix}
\NormalTok{thismatrix}
\end{Highlighting}
\end{Shaded}

\begin{verbatim}
##      [,1] [,2]
## [1,]    1    4
## [2,]    2    5
## [3,]    3    6
\end{verbatim}

\begin{Shaded}
\begin{Highlighting}[]
\CommentTok{\# Access matrix items}
\NormalTok{thismatrix }\OtherTok{\textless{}{-}} \FunctionTok{matrix}\NormalTok{(}\FunctionTok{c}\NormalTok{(}\StringTok{"apple"}\NormalTok{, }\StringTok{"banana"}\NormalTok{, }\StringTok{"cherry"}\NormalTok{, }\StringTok{"orange"}\NormalTok{), }\AttributeTok{nrow =} \DecValTok{2}\NormalTok{, }\AttributeTok{ncol =} \DecValTok{2}\NormalTok{)}

\NormalTok{thismatrix[}\DecValTok{1}\NormalTok{, }\DecValTok{2}\NormalTok{]}
\end{Highlighting}
\end{Shaded}

\begin{verbatim}
## [1] "cherry"
\end{verbatim}

\begin{Shaded}
\begin{Highlighting}[]
\CommentTok{\# Access more than one row}
\NormalTok{thismatrix }\OtherTok{\textless{}{-}} \FunctionTok{matrix}\NormalTok{(}\FunctionTok{c}\NormalTok{(}\StringTok{"apple"}\NormalTok{, }\StringTok{"banana"}\NormalTok{, }\StringTok{"cherry"}\NormalTok{, }\StringTok{"orange"}\NormalTok{,}\StringTok{"grape"}\NormalTok{, }\StringTok{"pineapple"}\NormalTok{, }\StringTok{"pear"}\NormalTok{, }\StringTok{"melon"}\NormalTok{, }\StringTok{"fig"}\NormalTok{), }\AttributeTok{nrow =} \DecValTok{3}\NormalTok{, }\AttributeTok{ncol =} \DecValTok{3}\NormalTok{)}

\NormalTok{thismatrix[}\FunctionTok{c}\NormalTok{(}\DecValTok{1}\NormalTok{,}\DecValTok{2}\NormalTok{),]}
\end{Highlighting}
\end{Shaded}

\begin{verbatim}
##      [,1]     [,2]     [,3]   
## [1,] "apple"  "orange" "pear" 
## [2,] "banana" "grape"  "melon"
\end{verbatim}

\begin{Shaded}
\begin{Highlighting}[]
\CommentTok{\# Access more than one column}
\NormalTok{thismatrix }\OtherTok{\textless{}{-}} \FunctionTok{matrix}\NormalTok{(}\FunctionTok{c}\NormalTok{(}\StringTok{"apple"}\NormalTok{, }\StringTok{"banana"}\NormalTok{, }\StringTok{"cherry"}\NormalTok{, }\StringTok{"orange"}\NormalTok{,}\StringTok{"grape"}\NormalTok{, }\StringTok{"pineapple"}\NormalTok{, }\StringTok{"pear"}\NormalTok{, }\StringTok{"melon"}\NormalTok{, }\StringTok{"fig"}\NormalTok{), }\AttributeTok{nrow =} \DecValTok{3}\NormalTok{, }\AttributeTok{ncol =} \DecValTok{3}\NormalTok{)}

\NormalTok{thismatrix[, }\FunctionTok{c}\NormalTok{(}\DecValTok{1}\NormalTok{,}\DecValTok{2}\NormalTok{)]}
\end{Highlighting}
\end{Shaded}

\begin{verbatim}
##      [,1]     [,2]       
## [1,] "apple"  "orange"   
## [2,] "banana" "grape"    
## [3,] "cherry" "pineapple"
\end{verbatim}

\begin{Shaded}
\begin{Highlighting}[]
\CommentTok{\# Add rows and columns}
\NormalTok{newmatrix }\OtherTok{\textless{}{-}} \FunctionTok{cbind}\NormalTok{(thismatrix, }\FunctionTok{c}\NormalTok{(}\StringTok{"strawberry"}\NormalTok{, }\StringTok{"blueberry"}\NormalTok{, }\StringTok{"raspberry"}\NormalTok{))}

\CommentTok{\# Print the new matrix}
\NormalTok{newmatrix}
\end{Highlighting}
\end{Shaded}

\begin{verbatim}
##      [,1]     [,2]        [,3]    [,4]        
## [1,] "apple"  "orange"    "pear"  "strawberry"
## [2,] "banana" "grape"     "melon" "blueberry" 
## [3,] "cherry" "pineapple" "fig"   "raspberry"
\end{verbatim}

\begin{Shaded}
\begin{Highlighting}[]
\CommentTok{\#Remove the first row and the first column}
\NormalTok{thismatrix }\OtherTok{\textless{}{-}}\NormalTok{ thismatrix[}\SpecialCharTok{{-}}\FunctionTok{c}\NormalTok{(}\DecValTok{1}\NormalTok{), }\SpecialCharTok{{-}}\FunctionTok{c}\NormalTok{(}\DecValTok{1}\NormalTok{)]}

\NormalTok{thismatrix}
\end{Highlighting}
\end{Shaded}

\begin{verbatim}
##      [,1]        [,2]   
## [1,] "grape"     "melon"
## [2,] "pineapple" "fig"
\end{verbatim}

\begin{Shaded}
\begin{Highlighting}[]
\CommentTok{\# Check if items exist}
\NormalTok{thismatrix }\OtherTok{\textless{}{-}} \FunctionTok{matrix}\NormalTok{(}\FunctionTok{c}\NormalTok{(}\StringTok{"apple"}\NormalTok{, }\StringTok{"banana"}\NormalTok{, }\StringTok{"cherry"}\NormalTok{, }\StringTok{"orange"}\NormalTok{), }\AttributeTok{nrow =} \DecValTok{2}\NormalTok{, }\AttributeTok{ncol =} \DecValTok{2}\NormalTok{)}

\StringTok{"apple"} \SpecialCharTok{\%in\%}\NormalTok{ thismatrix}
\end{Highlighting}
\end{Shaded}

\begin{verbatim}
## [1] TRUE
\end{verbatim}

\begin{Shaded}
\begin{Highlighting}[]
\CommentTok{\# Matrix Length}
\FunctionTok{length}\NormalTok{(thismatrix)}
\end{Highlighting}
\end{Shaded}

\begin{verbatim}
## [1] 4
\end{verbatim}

\begin{Shaded}
\begin{Highlighting}[]
\CommentTok{\# Loop trough list}
\ControlFlowTok{for}\NormalTok{ (rows }\ControlFlowTok{in} \DecValTok{1}\SpecialCharTok{:}\FunctionTok{nrow}\NormalTok{(thismatrix)) \{}
  \ControlFlowTok{for}\NormalTok{ (columns }\ControlFlowTok{in} \DecValTok{1}\SpecialCharTok{:}\FunctionTok{ncol}\NormalTok{(thismatrix)) \{}
    \FunctionTok{print}\NormalTok{(thismatrix[rows, columns])}
\NormalTok{  \}}
\NormalTok{\}}
\end{Highlighting}
\end{Shaded}

\begin{verbatim}
## [1] "apple"
## [1] "cherry"
## [1] "banana"
## [1] "orange"
\end{verbatim}

\begin{Shaded}
\begin{Highlighting}[]
\CommentTok{\# Combine two matrices}
\CommentTok{\# Combine matrices}
\NormalTok{Matrix1 }\OtherTok{\textless{}{-}} \FunctionTok{matrix}\NormalTok{(}\FunctionTok{c}\NormalTok{(}\StringTok{"apple"}\NormalTok{, }\StringTok{"banana"}\NormalTok{, }\StringTok{"cherry"}\NormalTok{, }\StringTok{"grape"}\NormalTok{), }\AttributeTok{nrow =} \DecValTok{2}\NormalTok{, }\AttributeTok{ncol =} \DecValTok{2}\NormalTok{)}
\NormalTok{Matrix2 }\OtherTok{\textless{}{-}} \FunctionTok{matrix}\NormalTok{(}\FunctionTok{c}\NormalTok{(}\StringTok{"orange"}\NormalTok{, }\StringTok{"mango"}\NormalTok{, }\StringTok{"pineapple"}\NormalTok{, }\StringTok{"watermelon"}\NormalTok{), }\AttributeTok{nrow =} \DecValTok{2}\NormalTok{, }\AttributeTok{ncol =} \DecValTok{2}\NormalTok{)}

\CommentTok{\# Adding it as a rows}
\NormalTok{Matrix\_Combined }\OtherTok{\textless{}{-}} \FunctionTok{rbind}\NormalTok{(Matrix1, Matrix2)}
\NormalTok{Matrix\_Combined}
\end{Highlighting}
\end{Shaded}

\begin{verbatim}
##      [,1]     [,2]        
## [1,] "apple"  "cherry"    
## [2,] "banana" "grape"     
## [3,] "orange" "pineapple" 
## [4,] "mango"  "watermelon"
\end{verbatim}

\begin{Shaded}
\begin{Highlighting}[]
\CommentTok{\# Adding it as a columns}
\NormalTok{Matrix\_Combined }\OtherTok{\textless{}{-}} \FunctionTok{cbind}\NormalTok{(Matrix1, Matrix2)}
\NormalTok{Matrix\_Combined}
\end{Highlighting}
\end{Shaded}

\begin{verbatim}
##      [,1]     [,2]     [,3]     [,4]        
## [1,] "apple"  "cherry" "orange" "pineapple" 
## [2,] "banana" "grape"  "mango"  "watermelon"
\end{verbatim}

\hypertarget{r-arrays}{%
\paragraph{R Arrays:}\label{r-arrays}}

\begin{Shaded}
\begin{Highlighting}[]
\CommentTok{\# Vector of strings}
\NormalTok{fruits }\OtherTok{\textless{}{-}} \FunctionTok{c}\NormalTok{(}\StringTok{"banana"}\NormalTok{, }\StringTok{"apple"}\NormalTok{, }\StringTok{"orange"}\NormalTok{)}

\CommentTok{\# Print fruits}
\NormalTok{fruits}
\end{Highlighting}
\end{Shaded}

\begin{verbatim}
## [1] "banana" "apple"  "orange"
\end{verbatim}

\begin{Shaded}
\begin{Highlighting}[]
\CommentTok{\# Vector of numerical values}
\NormalTok{numbers }\OtherTok{\textless{}{-}} \FunctionTok{c}\NormalTok{(}\DecValTok{1}\NormalTok{, }\DecValTok{2}\NormalTok{, }\DecValTok{3}\NormalTok{)}

\CommentTok{\# Print numbers}
\NormalTok{numbers}
\end{Highlighting}
\end{Shaded}

\begin{verbatim}
## [1] 1 2 3
\end{verbatim}

\begin{Shaded}
\begin{Highlighting}[]
\CommentTok{\# Vector with numerical values in a sequence}
\NormalTok{numbers }\OtherTok{\textless{}{-}} \DecValTok{1}\SpecialCharTok{:}\DecValTok{10}

\NormalTok{numbers}
\end{Highlighting}
\end{Shaded}

\begin{verbatim}
##  [1]  1  2  3  4  5  6  7  8  9 10
\end{verbatim}

\begin{Shaded}
\begin{Highlighting}[]
\CommentTok{\# Length of vectors:}
\NormalTok{fruits }\OtherTok{\textless{}{-}} \FunctionTok{c}\NormalTok{(}\StringTok{"banana"}\NormalTok{, }\StringTok{"apple"}\NormalTok{, }\StringTok{"orange"}\NormalTok{)}

\FunctionTok{length}\NormalTok{(fruits)}
\end{Highlighting}
\end{Shaded}

\begin{verbatim}
## [1] 3
\end{verbatim}

\begin{Shaded}
\begin{Highlighting}[]
\CommentTok{\# Sort a vector:}
\NormalTok{fruits }\OtherTok{\textless{}{-}} \FunctionTok{c}\NormalTok{(}\StringTok{"banana"}\NormalTok{, }\StringTok{"apple"}\NormalTok{, }\StringTok{"orange"}\NormalTok{, }\StringTok{"mango"}\NormalTok{, }\StringTok{"lemon"}\NormalTok{)}
\NormalTok{numbers }\OtherTok{\textless{}{-}} \FunctionTok{c}\NormalTok{(}\DecValTok{13}\NormalTok{, }\DecValTok{3}\NormalTok{, }\DecValTok{5}\NormalTok{, }\DecValTok{7}\NormalTok{, }\DecValTok{20}\NormalTok{, }\DecValTok{2}\NormalTok{)}

\FunctionTok{sort}\NormalTok{(fruits)  }\CommentTok{\# Sort a string}
\end{Highlighting}
\end{Shaded}

\begin{verbatim}
## [1] "apple"  "banana" "lemon"  "mango"  "orange"
\end{verbatim}

\begin{Shaded}
\begin{Highlighting}[]
\FunctionTok{sort}\NormalTok{(numbers) }\CommentTok{\# Sort numbers}
\end{Highlighting}
\end{Shaded}

\begin{verbatim}
## [1]  2  3  5  7 13 20
\end{verbatim}

\begin{Shaded}
\begin{Highlighting}[]
\CommentTok{\# Accessing vectors:}
\NormalTok{fruits[}\DecValTok{1}\NormalTok{]}
\end{Highlighting}
\end{Shaded}

\begin{verbatim}
## [1] "banana"
\end{verbatim}

\begin{Shaded}
\begin{Highlighting}[]
\CommentTok{\# Change "banana" to "pear"}
\NormalTok{fruits[}\DecValTok{1}\NormalTok{] }\OtherTok{\textless{}{-}} \StringTok{"pear"}

\CommentTok{\# Print fruits}
\NormalTok{fruits}
\end{Highlighting}
\end{Shaded}

\begin{verbatim}
## [1] "pear"   "apple"  "orange" "mango"  "lemon"
\end{verbatim}

\begin{Shaded}
\begin{Highlighting}[]
\CommentTok{\# Repeat vectors:}
\NormalTok{repeat\_each }\OtherTok{\textless{}{-}} \FunctionTok{rep}\NormalTok{(}\FunctionTok{c}\NormalTok{(}\DecValTok{1}\NormalTok{,}\DecValTok{2}\NormalTok{,}\DecValTok{3}\NormalTok{), }\AttributeTok{each =} \DecValTok{3}\NormalTok{)}

\NormalTok{repeat\_each}
\end{Highlighting}
\end{Shaded}

\begin{verbatim}
## [1] 1 1 1 2 2 2 3 3 3
\end{verbatim}

\begin{Shaded}
\begin{Highlighting}[]
\NormalTok{numbers }\OtherTok{\textless{}{-}} \FunctionTok{seq}\NormalTok{(}\AttributeTok{from =} \DecValTok{0}\NormalTok{, }\AttributeTok{to =} \DecValTok{100}\NormalTok{, }\AttributeTok{by =} \DecValTok{20}\NormalTok{)}

\NormalTok{numbers}
\end{Highlighting}
\end{Shaded}

\begin{verbatim}
## [1]   0  20  40  60  80 100
\end{verbatim}

\hypertarget{r-lists-1}{%
\paragraph{R Lists:}\label{r-lists-1}}

\begin{Shaded}
\begin{Highlighting}[]
\CommentTok{\# List of strings}
\NormalTok{thislist }\OtherTok{\textless{}{-}} \FunctionTok{list}\NormalTok{(}\StringTok{"apple"}\NormalTok{, }\StringTok{"banana"}\NormalTok{, }\StringTok{"cherry"}\NormalTok{)}

\CommentTok{\# Print the list}
\NormalTok{thislist}
\end{Highlighting}
\end{Shaded}

\begin{verbatim}
## [[1]]
## [1] "apple"
## 
## [[2]]
## [1] "banana"
## 
## [[3]]
## [1] "cherry"
\end{verbatim}

\begin{Shaded}
\begin{Highlighting}[]
\CommentTok{\# Access the list}
\NormalTok{thislist[}\DecValTok{1}\NormalTok{]}
\end{Highlighting}
\end{Shaded}

\begin{verbatim}
## [[1]]
## [1] "apple"
\end{verbatim}

\begin{Shaded}
\begin{Highlighting}[]
\CommentTok{\# Change item in list}
\NormalTok{thislist }\OtherTok{\textless{}{-}} \FunctionTok{list}\NormalTok{(}\StringTok{"apple"}\NormalTok{, }\StringTok{"banana"}\NormalTok{, }\StringTok{"cherry"}\NormalTok{)}
\NormalTok{thislist[}\DecValTok{1}\NormalTok{] }\OtherTok{\textless{}{-}} \StringTok{"blackcurrant"}

\CommentTok{\# Length of a list}
\FunctionTok{length}\NormalTok{(thislist)}
\end{Highlighting}
\end{Shaded}

\begin{verbatim}
## [1] 3
\end{verbatim}

\begin{Shaded}
\begin{Highlighting}[]
\CommentTok{\# Check if item exists}
\StringTok{"apple"} \SpecialCharTok{\%in\%}\NormalTok{ thislist}
\end{Highlighting}
\end{Shaded}

\begin{verbatim}
## [1] FALSE
\end{verbatim}

\begin{Shaded}
\begin{Highlighting}[]
\CommentTok{\# Add item to list}
\FunctionTok{append}\NormalTok{(thislist, }\StringTok{"orange"}\NormalTok{)}
\end{Highlighting}
\end{Shaded}

\begin{verbatim}
## [[1]]
## [1] "blackcurrant"
## 
## [[2]]
## [1] "banana"
## 
## [[3]]
## [1] "cherry"
## 
## [[4]]
## [1] "orange"
\end{verbatim}

\begin{Shaded}
\begin{Highlighting}[]
\CommentTok{\# Remove item from list}
\NormalTok{newlist }\OtherTok{\textless{}{-}}\NormalTok{ thislist[}\SpecialCharTok{{-}}\DecValTok{1}\NormalTok{]}

\CommentTok{\# Range of indexes}
\NormalTok{(thislist)[}\DecValTok{2}\SpecialCharTok{:}\DecValTok{5}\NormalTok{]}
\end{Highlighting}
\end{Shaded}

\begin{verbatim}
## [[1]]
## [1] "banana"
## 
## [[2]]
## [1] "cherry"
## 
## [[3]]
## NULL
## 
## [[4]]
## NULL
\end{verbatim}

\begin{Shaded}
\begin{Highlighting}[]
\CommentTok{\# Loop through list}
\ControlFlowTok{for}\NormalTok{ (x }\ControlFlowTok{in}\NormalTok{ thislist) \{}
  \FunctionTok{print}\NormalTok{(x)}
\NormalTok{\}}
\end{Highlighting}
\end{Shaded}

\begin{verbatim}
## [1] "blackcurrant"
## [1] "banana"
## [1] "cherry"
\end{verbatim}

\begin{Shaded}
\begin{Highlighting}[]
\CommentTok{\# Join two lists}
\NormalTok{list1 }\OtherTok{\textless{}{-}} \FunctionTok{list}\NormalTok{(}\StringTok{"a"}\NormalTok{, }\StringTok{"b"}\NormalTok{, }\StringTok{"c"}\NormalTok{)}
\NormalTok{list2 }\OtherTok{\textless{}{-}} \FunctionTok{list}\NormalTok{(}\DecValTok{1}\NormalTok{,}\DecValTok{2}\NormalTok{,}\DecValTok{3}\NormalTok{)}
\NormalTok{list3 }\OtherTok{\textless{}{-}} \FunctionTok{c}\NormalTok{(list1,list2)}

\NormalTok{list3}
\end{Highlighting}
\end{Shaded}

\begin{verbatim}
## [[1]]
## [1] "a"
## 
## [[2]]
## [1] "b"
## 
## [[3]]
## [1] "c"
## 
## [[4]]
## [1] 1
## 
## [[5]]
## [1] 2
## 
## [[6]]
## [1] 3
\end{verbatim}

\hypertarget{r-matrices-1}{%
\paragraph{R Matrices:}\label{r-matrices-1}}

\begin{Shaded}
\begin{Highlighting}[]
\CommentTok{\# An array with one dimension with values ranging from 1 to 24}
\NormalTok{thisarray }\OtherTok{\textless{}{-}} \FunctionTok{c}\NormalTok{(}\DecValTok{1}\SpecialCharTok{:}\DecValTok{24}\NormalTok{)}
\NormalTok{thisarray}
\end{Highlighting}
\end{Shaded}

\begin{verbatim}
##  [1]  1  2  3  4  5  6  7  8  9 10 11 12 13 14 15 16 17 18 19 20 21 22 23 24
\end{verbatim}

\begin{Shaded}
\begin{Highlighting}[]
\CommentTok{\# An array with more than one dimension}
\NormalTok{multiarray }\OtherTok{\textless{}{-}} \FunctionTok{array}\NormalTok{(thisarray, }\AttributeTok{dim =} \FunctionTok{c}\NormalTok{(}\DecValTok{4}\NormalTok{, }\DecValTok{3}\NormalTok{, }\DecValTok{2}\NormalTok{))}
\NormalTok{multiarray}
\end{Highlighting}
\end{Shaded}

\begin{verbatim}
## , , 1
## 
##      [,1] [,2] [,3]
## [1,]    1    5    9
## [2,]    2    6   10
## [3,]    3    7   11
## [4,]    4    8   12
## 
## , , 2
## 
##      [,1] [,2] [,3]
## [1,]   13   17   21
## [2,]   14   18   22
## [3,]   15   19   23
## [4,]   16   20   24
\end{verbatim}

\begin{Shaded}
\begin{Highlighting}[]
\CommentTok{\# Access Array}
\NormalTok{multiarray[}\DecValTok{2}\NormalTok{, }\DecValTok{3}\NormalTok{, }\DecValTok{2}\NormalTok{]}
\end{Highlighting}
\end{Shaded}

\begin{verbatim}
## [1] 22
\end{verbatim}

\begin{Shaded}
\begin{Highlighting}[]
\CommentTok{\# Check if item exists}
\DecValTok{2} \SpecialCharTok{\%in\%}\NormalTok{ multiarray}
\end{Highlighting}
\end{Shaded}

\begin{verbatim}
## [1] TRUE
\end{verbatim}

\begin{Shaded}
\begin{Highlighting}[]
\CommentTok{\# Amount of rows}
\FunctionTok{dim}\NormalTok{(multiarray)}
\end{Highlighting}
\end{Shaded}

\begin{verbatim}
## [1] 4 3 2
\end{verbatim}

\begin{Shaded}
\begin{Highlighting}[]
\CommentTok{\# Array Length}
\FunctionTok{length}\NormalTok{(multiarray)}
\end{Highlighting}
\end{Shaded}

\begin{verbatim}
## [1] 24
\end{verbatim}

\begin{Shaded}
\begin{Highlighting}[]
\CommentTok{\# Loop trough array}
\ControlFlowTok{for}\NormalTok{(x }\ControlFlowTok{in}\NormalTok{ multiarray)\{}
  \FunctionTok{print}\NormalTok{(x)}
\NormalTok{\}}
\end{Highlighting}
\end{Shaded}

\begin{verbatim}
## [1] 1
## [1] 2
## [1] 3
## [1] 4
## [1] 5
## [1] 6
## [1] 7
## [1] 8
## [1] 9
## [1] 10
## [1] 11
## [1] 12
## [1] 13
## [1] 14
## [1] 15
## [1] 16
## [1] 17
## [1] 18
## [1] 19
## [1] 20
## [1] 21
## [1] 22
## [1] 23
## [1] 24
\end{verbatim}

\hypertarget{r-data-frames}{%
\paragraph{R Data Frames:}\label{r-data-frames}}

\begin{Shaded}
\begin{Highlighting}[]
\CommentTok{\# Create a data frame}
\NormalTok{Data\_Frame }\OtherTok{\textless{}{-}} \FunctionTok{data.frame}\NormalTok{ (}
  \AttributeTok{Training =} \FunctionTok{c}\NormalTok{(}\StringTok{"Strength"}\NormalTok{, }\StringTok{"Stamina"}\NormalTok{, }\StringTok{"Other"}\NormalTok{),}
  \AttributeTok{Pulse =} \FunctionTok{c}\NormalTok{(}\DecValTok{100}\NormalTok{, }\DecValTok{150}\NormalTok{, }\DecValTok{120}\NormalTok{),}
  \AttributeTok{Duration =} \FunctionTok{c}\NormalTok{(}\DecValTok{60}\NormalTok{, }\DecValTok{30}\NormalTok{, }\DecValTok{45}\NormalTok{)}
\NormalTok{)}

\CommentTok{\# Print the data frame}
\NormalTok{Data\_Frame}
\end{Highlighting}
\end{Shaded}

\begin{verbatim}
##   Training Pulse Duration
## 1 Strength   100       60
## 2  Stamina   150       30
## 3    Other   120       45
\end{verbatim}

\begin{Shaded}
\begin{Highlighting}[]
\CommentTok{\# Summarize Data}
\FunctionTok{summary}\NormalTok{(Data\_Frame)}
\end{Highlighting}
\end{Shaded}

\begin{verbatim}
##    Training             Pulse          Duration   
##  Length:3           Min.   :100.0   Min.   :30.0  
##  Class :character   1st Qu.:110.0   1st Qu.:37.5  
##  Mode  :character   Median :120.0   Median :45.0  
##                     Mean   :123.3   Mean   :45.0  
##                     3rd Qu.:135.0   3rd Qu.:52.5  
##                     Max.   :150.0   Max.   :60.0
\end{verbatim}

\begin{Shaded}
\begin{Highlighting}[]
\CommentTok{\# Access items}
\NormalTok{Data\_Frame[}\DecValTok{1}\NormalTok{]}
\end{Highlighting}
\end{Shaded}

\begin{verbatim}
##   Training
## 1 Strength
## 2  Stamina
## 3    Other
\end{verbatim}

\begin{Shaded}
\begin{Highlighting}[]
\NormalTok{Data\_Frame[[}\StringTok{"Training"}\NormalTok{]]}
\end{Highlighting}
\end{Shaded}

\begin{verbatim}
## [1] "Strength" "Stamina"  "Other"
\end{verbatim}

\begin{Shaded}
\begin{Highlighting}[]
\NormalTok{Data\_Frame}\SpecialCharTok{$}\NormalTok{Training}
\end{Highlighting}
\end{Shaded}

\begin{verbatim}
## [1] "Strength" "Stamina"  "Other"
\end{verbatim}

\begin{Shaded}
\begin{Highlighting}[]
\CommentTok{\# Add a new row}
\NormalTok{New\_row\_DF }\OtherTok{\textless{}{-}} \FunctionTok{rbind}\NormalTok{(Data\_Frame, }\FunctionTok{c}\NormalTok{(}\StringTok{"Strength"}\NormalTok{, }\DecValTok{110}\NormalTok{, }\DecValTok{110}\NormalTok{))}

\CommentTok{\# Add a new column}
\NormalTok{New\_col\_DF }\OtherTok{\textless{}{-}} \FunctionTok{cbind}\NormalTok{(Data\_Frame, }\AttributeTok{Steps =} \FunctionTok{c}\NormalTok{(}\DecValTok{1000}\NormalTok{, }\DecValTok{6000}\NormalTok{, }\DecValTok{2000}\NormalTok{))}

\CommentTok{\# Remove the first row and column}
\NormalTok{Data\_Frame\_New }\OtherTok{\textless{}{-}}\NormalTok{ Data\_Frame[}\SpecialCharTok{{-}}\FunctionTok{c}\NormalTok{(}\DecValTok{1}\NormalTok{), }\SpecialCharTok{{-}}\FunctionTok{c}\NormalTok{(}\DecValTok{1}\NormalTok{)]}

\CommentTok{\# Data Frame length}
\FunctionTok{length}\NormalTok{(Data\_Frame)}
\end{Highlighting}
\end{Shaded}

\begin{verbatim}
## [1] 3
\end{verbatim}

\begin{Shaded}
\begin{Highlighting}[]
\CommentTok{\# Combining Data Frames}
\NormalTok{Data\_Frame1 }\OtherTok{\textless{}{-}} \FunctionTok{data.frame}\NormalTok{ (}
  \AttributeTok{Training =} \FunctionTok{c}\NormalTok{(}\StringTok{"Strength"}\NormalTok{, }\StringTok{"Stamina"}\NormalTok{, }\StringTok{"Other"}\NormalTok{),}
  \AttributeTok{Pulse =} \FunctionTok{c}\NormalTok{(}\DecValTok{100}\NormalTok{, }\DecValTok{150}\NormalTok{, }\DecValTok{120}\NormalTok{),}
  \AttributeTok{Duration =} \FunctionTok{c}\NormalTok{(}\DecValTok{60}\NormalTok{, }\DecValTok{30}\NormalTok{, }\DecValTok{45}\NormalTok{)}
\NormalTok{)}

\NormalTok{Data\_Frame2 }\OtherTok{\textless{}{-}} \FunctionTok{data.frame}\NormalTok{ (}
  \AttributeTok{Training =} \FunctionTok{c}\NormalTok{(}\StringTok{"Stamina"}\NormalTok{, }\StringTok{"Stamina"}\NormalTok{, }\StringTok{"Strength"}\NormalTok{),}
  \AttributeTok{Pulse =} \FunctionTok{c}\NormalTok{(}\DecValTok{140}\NormalTok{, }\DecValTok{150}\NormalTok{, }\DecValTok{160}\NormalTok{),}
  \AttributeTok{Duration =} \FunctionTok{c}\NormalTok{(}\DecValTok{30}\NormalTok{, }\DecValTok{30}\NormalTok{, }\DecValTok{20}\NormalTok{)}
\NormalTok{)}

\NormalTok{New\_Data\_Frame }\OtherTok{\textless{}{-}} \FunctionTok{rbind}\NormalTok{(Data\_Frame1, Data\_Frame2)}
\NormalTok{New\_Data\_Frame}
\end{Highlighting}
\end{Shaded}

\begin{verbatim}
##   Training Pulse Duration
## 1 Strength   100       60
## 2  Stamina   150       30
## 3    Other   120       45
## 4  Stamina   140       30
## 5  Stamina   150       30
## 6 Strength   160       20
\end{verbatim}

\hypertarget{r-factors}{%
\paragraph{R Factors:}\label{r-factors}}

\begin{Shaded}
\begin{Highlighting}[]
\CommentTok{\# Create a factor}
\NormalTok{music\_genre }\OtherTok{\textless{}{-}} \FunctionTok{factor}\NormalTok{(}\FunctionTok{c}\NormalTok{(}\StringTok{"Jazz"}\NormalTok{, }\StringTok{"Rock"}\NormalTok{, }\StringTok{"Classic"}\NormalTok{, }\StringTok{"Classic"}\NormalTok{, }\StringTok{"Pop"}\NormalTok{, }\StringTok{"Jazz"}\NormalTok{, }\StringTok{"Rock"}\NormalTok{, }\StringTok{"Jazz"}\NormalTok{))}

\CommentTok{\# Print the factor}
\NormalTok{music\_genre}
\end{Highlighting}
\end{Shaded}

\begin{verbatim}
## [1] Jazz    Rock    Classic Classic Pop     Jazz    Rock    Jazz   
## Levels: Classic Jazz Pop Rock
\end{verbatim}

\begin{Shaded}
\begin{Highlighting}[]
\CommentTok{\# Factor length}
\FunctionTok{length}\NormalTok{(music\_genre)}
\end{Highlighting}
\end{Shaded}

\begin{verbatim}
## [1] 8
\end{verbatim}

\begin{Shaded}
\begin{Highlighting}[]
\CommentTok{\# Access Factors}
\NormalTok{music\_genre[}\DecValTok{3}\NormalTok{]}
\end{Highlighting}
\end{Shaded}

\begin{verbatim}
## [1] Classic
## Levels: Classic Jazz Pop Rock
\end{verbatim}

\begin{Shaded}
\begin{Highlighting}[]
\CommentTok{\# Chage item value}
\NormalTok{music\_genre[}\DecValTok{3}\NormalTok{] }\OtherTok{\textless{}{-}} \StringTok{"Pop"}
\end{Highlighting}
\end{Shaded}

Note that the \texttt{echo\ =\ FALSE} parameter was added to the code
chunk to prevent printing of the R code that generated the plot.

\end{document}
